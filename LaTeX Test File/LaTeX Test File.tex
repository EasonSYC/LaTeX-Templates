%!TEX TX-program = xelatex
\documentclass{article}
\usepackage{allan-eason}

\usetikzlibrary{positioning}
\usetikzlibrary{svg.path}

\graphicspath{ {./images/}} 

\newcommand{\Title}{\LaTeX\ Test File}
\newcommand{\Author}{Eason S.}

\title{\Title}
\author{\Author}
\date{\today}

\geometry{a4paper, scale=0.8}

\lhead{\Title}

\begin{document}

	\maketitle

	\section{Maxwell's Equations}

		\subsection{Integral Format}

			\defword{Maxwell's Equations} (in forms of \defword{Integral}):

			\begin{align}
				\oiint_{S} \vect{D} \cdot \diff \vect{S} = \sum q &= \int_V \rho \diff V,\\
				\oiint_{S} \vect{B} \cdot \diff \vect{S} &= 0,\\
				\oint_{L} \vect{H} \cdot \diff \vect{l} = I + I_{\diff} &= \int_{S} \vect{j} \cdot \diff \vect{S} + \int_{S} \frac{\partial \vect{D}}{\partial t} \cdot \diff \vect{S},\\
				\oint_{L} \vect{E} \cdot \diff \vect{l} = - \frac{\diff \varPhi}{\diff t} &= - \int_{S} \frac{\partial \vect{B}}{\partial t} \cdot \diff \vect{S}.
			\end{align}

			Here, (1) states for the \defword{Gauss Theorem} in an \defword{Electric Field}, while (2) states for the \defword{Gauss Theorem} in an \defword{Magnetic Field}. (3) states for the relationship between \defword{A Changing Electric Field} and a magnetic field, or \defword{Ampere's Circulation Theorem}. (4) states for the relationship between \defword{A Changing Magnetic Field} and a electric field, or \defword{Faraday's Theorem of induction}.

	\section{Partial Derivative}

		\subsection{Definition}

			Let \(t=f(x, y, \ldots)\), the \defword{Partial Derivative} of \(f\) towards \(x\) is

			\[
				f'_{x} = \partial_x f = D_x f = D_1 f = \frac{\partial}{\partial x} f = \frac{\partial f}{\partial x} = \lim_{\Delta x \rightarrow 0} \frac{f(x + \Delta x, y, \ldots) - f(x, y, \ldots)}{\Delta x}.
			\]

			Define vector \(\vect{a} = (x, y, \ldots), \vect{\hat{e}_x} = (1, 0, \ldots)\), therefore

			\[
				\frac{\partial}{\partial x} f = \lim_{x\rightarrow 0} \frac{f(\vect{a} + h \vect{e_x}) - f(\vect{a})}{h}.
			\]

		\subsection{Gradient}

			Define \defword{Gradient} as following:

			\[
				\Grad f(\vect{a}) = \nabla f(\vect{a}) = \left(\at{\frac{\partial f}{\partial x}}{\vect{a}}, \at{\frac{\partial f}{\partial y}}{\vect{a}}, \ldots\right).
			\]

			We usually deine Gradient as following in a 3-Dimensional Space:

			\[
				\Grad = \nabla = \left[\frac{\partial}{\partial x}\right]\vect{\hat{e}_x} + \left[\frac{\partial}{\partial y}\right]\vect{\hat{e}_y} + \left[\frac{\partial}{\partial z}\right]\vect{\hat{e}_z}.
			\]

		\subsection{Directional Derivative}

			Define the \defword{Directional Derivative} along vector \(\vect{v} = \left(v_1, v_2, \ldots\right)\),

			\[
				\nabla_{\vect{v}} f(\vect{a}) = \lim_{x \rightarrow 0} \frac{f(\vect{a} + h \vect{v}) - f(\vect{a})}{h}.
			\]

		\subsection{Laplace Operator}

			Define the \defword{Laplace Operator} as following:

			\[
				\Delta = \frac{\partial^2}{\partial x^2} + \frac{\partial^2}{\partial y^2} + \frac{\partial^2}{\partial z^2} = \nabla \cdot \nabla = \nabla^2.
			\]

		\subsection{Divergence}

			Define the \defword{Divergence} of a vector as following: (it outputs a value)

			\[
				\Div \vect{v} = \nabla \cdot \vect{v} = \left(\frac{\partial}{\partial x}, \frac{\partial}{\partial y}, \frac{\partial}{\partial z} \right) \cdot \left(v_x, v_y, v_z\right) = \frac{\partial v_x}{\partial x} + \frac{\partial v_y}{\partial y} + \frac{\partial v_z}{\partial z}.
			\]


\end{document}